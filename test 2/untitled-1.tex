\documentclass{homework}
\usepackage{enumitem}
\usepackage{tikz}
\usepackage{pgfplots}

\title{Test 2}
\author{
  Dmitrii, Maksimov\\
  \texttt{maksimov.dmitrii.m@gmail.com}
}
\begin{document}

\maketitle

\exercise
\begin{enumerate}[label=\alph*)]
	\item $Q = \{x\in\R^n|\sum_{i=1}^nx_i^2=1 \text{ and } x_i \geq0\;\forall i=1,\ldots,n\}$\newline
		Let $x = (1, 0)^T, y=(0,1)^T \in Q \text{ and } z = (0.5, 0.5)^T \in \{\alpha x + (1-\alpha)y\}, \text{ where } \alpha=0.5$, then $z \notin Q$. This is because $\sum_{i=1}^2z_i\neq1$ - non-convex
	\item $Q = \R^n$ - convex
	\item $Q = \{x\in\R^n|\sum_{i=1}^nx_i<0\;\forall i=1,\ldots,n\}$\newline
		$\text{Let } x, y \in Q \text{ and } z = \alpha x + (1-\alpha)y, \text{ then }z\in Q.$\newline This is because $\sum_{i=1}^nz_i=\alpha\sum_{i=1}^nx_i + (1-\alpha)\sum_{i=1}^ny_i < 0$ - convex
	\item $Q=\{0\}$ - a single point is a convex set
\end{enumerate}
Answer: a
\exercise*
Answer: c
\exercise*
$f(x)$ - $L$-smooth and $\mu$-strongly convex
\begin{enumerate}[label=\alph*)]
	\item if $f(x)$ twice differentiable, then $\lambda_{min}(\nabla^2f(x))\geq\mu \text{ and }\lambda_{max}(\nabla^2f(x))\leq L $ \newline From lectures: $0\preceq\nabla^2f(x)\preceq LI_d \text{ and }\nabla^2f(x)\succeq \mu I_d$ - True
	\item It can be that $L=\frac{\mu}{2}$ - \textbf{the answer is False, but I don't understand, could you explain?}
	\item $\forall x,y\;f(y)\leq f(x) - \langle \nabla f(x), y-x\rangle + \frac{L}{2}||y-x||_2^2 \newline\text{ and } f(y)\geq f(x) - \langle \nabla f(x), y-x\rangle + \frac{\mu}{2}||y-x||_2^2$ - False $(+ \langle \nabla f(x), y-x\rangle)$
	\item $\forall x,y\;\langle \nabla f(y) - \nabla f(x), y-x\rangle\geq\frac{1}{\mu}||\nabla f(x)-\nabla f(y)||_2^2 \newline\text{ and } \langle \nabla f(y) - \nabla f(x), y-x\rangle\geq L||x-y||_2^2$. The first inequality should be $\frac{1}{L}$ rather than $\frac{1}{\mu}$ and the second one $\leq$ rather than $\geq$ - False.
\end{enumerate}
Answer: a
\exercise*
\begin{tikzpicture}
\begin{axis}[
    axis lines = left,
    xlabel = \(x\),
    ylabel = {\(f(x)\)},
]
\addplot [
    domain=-5:-3, 
    samples=10, 
]
{-3*x - 6};
\addplot [
    domain=-3:-1, 
    samples=10, 
    ]
    {-x};
\addplot [
    domain=-1:1, 
    samples=10, 
    ]
    {x^2};
\addplot [
    domain=1:3, 
    samples=10, 
    ]
    {x};
\addplot [
    domain=3:5, 
    samples=10, 
]
{3*x - 6};
\end{axis}
\end{tikzpicture}

The function is convex, but not strictly convex since there are linear parts which are non-strictly convex.

Answer: b
\exercise*
The function is $L$-smooth with $L=3 \Rightarrow \max\nabla f(x) = 3$. \textbf {However, I don't understand what "for L < 3 the function is not $L$-smooth" means. Could you explain it?}

Answer: d
\end{document}